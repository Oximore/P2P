
\section{Implémentation de la base de données}
\subsection{Fichiers}
Nous avons créé une base de données contenant la liste des fichiers possédés par les clients actuellement connectés. \\
Les fichiers sont représentés par une structure element. Elle contient entre autres champs un pointeur sur une chaîne de caractères \textit{key}, qui sera l'identifiant du fichier (la base pourra comporter plusiurs fichiers de même nom \textit{name} et même longueur \textit{length}, mais une clé sera unique dans la base de données). \\
Le champ \textit{peer\_list} pointe sur un objet de type \textit{struct list}, qui est ne liste chaînée de \textit{struct peer}. La structure \textit{struct peer} étant la structure que nous avons choisie pour stocker les différents utilisateurs connectés, \textit{peer\_list} représente la liste de pairs connectés possédant ce fichier.\\ 
La structure \textit{base}, qui représente notre base de données, est une liste chaînée d'\textit{element};\\
 
\begin{verbatim}

struct element
{
  struct element * next;
  char * key;
  char * name;
  int length;
  int p_size;
  struct list * peer_list; 
};

struct base
{
  struct element * first;
  int size;
};

\end{verbatim}

\subsection{Pairs}
  
Chaque pair est représenté par une structure \textit{peer}. Le champ \textit{time} permet de stocker la date de réception du dernier message d'``update'' du pair. Une autre fonction parcourera tous les pairs connectés en continu, et si la valeur de ce champ dépasse une valeur de time\_out préalablement fixée, ce pair sera supprimé.\\

\begin{verbatim}
struct peer
{
  struct peer * next;
  unsigned long ip_address;
  int port;
  int time;
  struct base * previous_update;
};

struct list
{
  struct peer * first;
  int size;
};
\end{verbatim}

Ces deux constructions sont en fait symétriques: un élément d'une des deux listes possède un champ du type de l'autre liste, ce qui laisse la possibilité de stocker les fichiers possédés dans les pairs connectés ou les pairs possédant le fichier dans les fichiers disponibles, voire les deux en stockant directement les pointeurs.\\ 
Par souci de simplicité nous avons décidé de stocker les pairs dans les fichiers.Toutefois le champ \textit{base} de la structure \textit{peer} nous sert quand même (cf Mise à jour).


\section{Les sockets}

Pour communiquer avec les pairs, le tracker doit créer des sockets. On crée aussi des threads associés à ces sockets.
\subsection{Gestion des threads et des sockets}
Pour stocker les threads et les sockets, on utilise une structure client et une structure client\_tab.

La structure client contient donc le file descriptor et la structure de la socket, ainsi que le thread associé :
\begin{verbatim}
struct client
{
  int sock;
  struct sockaddr_in *sockaddr;
  pthread_t *t; 
};
\end{verbatim}
On organise ensuite ces structures dans client\_tab :
\begin{verbatim}

struct client_tab
{
  struct client tab[MAX];
  int b[MAX];
};
\end{verbatim}
On utilise un tableau de longueur prédéfinie ainsi qu'un tableau de booléens de même taille. Ce tableau de booléens nous informe si les clients sont actifs ou déconnectés.
Nous avons choisi d'utiliser des tableaux car ils permettent de facilement limiter le nombre de connections au tracker ( pour qu'il ne soit pas surchargé )

\subsection{Thread d'écoute}
La socket d'écoute, créée sur le premier thread, sert à attendre les connections de nouveaux pairs (il est en attente bloquante). Dès qu'un pair demande une connection, ce premier thread crée une nouvelle socket et un nouveau thread spécifique à ce pair.
Le nouveau thread est lancé avec la fonction communicate.
\subsection{Threads de communication avec les pairs}
Ce thread de communication sert à recevoir et envoyer des messages au pair, et ce, au moyen de la fonction communicate. Cette fonction sera décrite ci-dessous.


\section{Communication avec le client}

La fonction \textit{communicate} écoute en permanence le client et stocke les messages reçus dans un buffer. Lorsqu'on a reçu des données, on parse ce buffer et on analyse les requètes du client. On effectue les mises à jour de la base de données si necessaires, puis on renvoie un message d'information ou de confirmation au client. \\
Si il y a une erreur (ex: le client envoie un message autre que ``announce ...'' alors qu'il n'est pas dans la liste des pairs connectés), on ferme la socket et on interromp la fonction, ce qui tue le processus. Ces fermetures peuvent aussi être faites par la fonction de time out si trop de temps s'est écoulé depuis le dernier ``update''.

\section{Mise à jour de la base de données}

A la réception d'un message d'''announce'', on crée un nouveau \textit{peer} s'il n'existe pas déjà (comparaison d'adresse ip). Puis, pour chaque fichier qu'il possède, on ajoute un \textit{element} à la base de données s'il n'existe pas déjà (comparaison de clés), sinon on ajoute ce pair à la liste des pairs possédant le fichier de même clé.\\

Lors de la reception d'un message d'''update'', on met à jour le champ time du pair, puis on compare la liste des fichiers possédés par le pair à ceux du précédent update (ils étaient stockés sous forme d'une liste d'\textit{element} dans le champ \textit{previous\_update} du pair).On met alors à jour la base de données en ajoutant le pair comme possesseur des fichiers qu'il a mais n'avait pas la dernière fois, et en l'enlevant des fichiers qu'il n'a plus. Puis on place la liste des fichiers de l'update dans le champ \textit{previous\_update} du pair. 
