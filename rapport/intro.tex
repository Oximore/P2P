% intro.tex

Ce projet a pour but de créer une application pair à pair (Peer to Peer). Pour cela, nous divisons l'application en deux parties : la partie tracker et
la partie pair. Les pairs sont des utilisateurs qui possèdent des
fichiers à partager et/ou qui désirent télécharger des fichiers
détenus par d'autres pairs. Pour cela, ils doivent se connecter au
tracker : ils renseignent ainsi le tracker sur les fichiers qu'ils possèdent et sur ceux qu'ils recherchent. Le tracker répond aux requêtes de téléchargement en
donnant la liste des pairs possédant le fichier désiré. Grâce à ces
informations, le pair peut alors commencer le téléchargement.  Le rôle
du tracker est donc central, on parle d'application centralisée.

Dans ce projet, la partie pair est réalisée en Java alors que la
partie tracker est réalisée en C.

