% intro.tex

Ce projet a pour but de créer une application pair à pair. Pour cela,
il faut diviser l'application en deux parties : la partie tracker et
la partie pair.  Les pair sont des utilisateurs qui possèdent des
fichiers à partager et/ou qui désirent télécharger des fichiers
détenus par d'autres pairs. Pour cela, ils doivent se connecter au
tracker : ils renseignent ainsi les fichiers qu'ils possèdent et ceux
qu'ils veulent. Le tracker répond aux requêtes de téléchargement en
donnant la liste des pairs possédant le fichier désiré. Grâce à ces
informations, le pair peut alors commencer le téléchargement.  Le rôle
du tracker est donc central, on parle d'application centralisée.

Dans ce projet, la partie paire est réalisée en Java alors que la
partie tracker est réalisée en C.

