Le pair est l'élément de base du système pair à pair. L'utilisateur de cette technologie stock des fichiers qu'il désire partager dans un dossier prévu à cet effet. Il lance ensuite son application (ici dans un terminal) puis il peut ensuite demander des informations sur d'autres fichiers possédés par d'autres pairs, ou bien le téléchargement de ceux ci.


\section{Structure générale}
Nous allons tout d'abord détailler la partie actuellement implémentée.
La gestion des connexion à travers différentes socket implique notamment la création d'une multitude de threads. Nous allons détailler chacun d'entre eux.

% Ces parties marchaient indépendamment
% Pourquoi cet ordre ?
% a faire : terminer tous les threads ... proprement ?
% lecture possible par un humain


\subsection{Thread principal}
Le thread principal (ou \textit{main}) est celui créé au lancement du programme. Ici son rôle va être de lancer les autres threads. 
Il va tout d'abord créer une structure de données, un objet de la classe \textit{FichierConfiguration}. Cet objet, à son instanciation, charge les données contenues dans le fichier désigné, ici \textit{configuration.txt}. 

Ensuite ce thread crée une table de hachage d'objet \textit{Fichier} indexée par les clés de ces fichiers. Une table de hachage est une structure de données adéquate pour le cas présent. En effet nous devons déjà hacher les fichiers, nous utilisons donc juste ces clés pour représenter les fichiers. 

Enfin, il lance un thread chargé de répondre au demandes des autres threads.

\subsection{Thread serveur du pair}
Ce thread commence par créer une socket serveur. Le port d'écoute est pour le moment choisit par défaut. Il faudra par la suite le déclarer dans les valeurs demandées (entre 60000 et 60025). 

Connaissant ensuite ce port grace à la méthode \textit{getPort()} il lance le thread qui préviendra le tracker de sa présence.

Il fini enfin par faire une boucle infinie qui attend les connexions entrantes. Celles-ci sont selon l'usage courant demandées par les autres pairs (le tracker ne demande jamais de connexion à un pair). Si une demande est reçue alors il crée un nouveau thread qui se chargera de répondre à cette demande, il continuera ensuite d'écouter sur son port.

\subsection{Thread des mise-à-jour}
C'est le premier appelé par le précédent. Il se charge de se connecter au serveur du tracker. Une fois ceci fait il déclare sa présence grace à un message $announce$ complété des informations contenues dans la table de hachage. 
Une fois la réponse $ok$ du tracker obtenue il ferme la connexion et lance une boucle infinie qui toute les $n$ minutes se reconnectera au tracker et le préviendra de l'actualité de ses fichiers. 

\subsection{Thread de réponse à un paire}
Ce thread est instancié avec une socket ouverte avec le pair qui a effectué la demande ainsi qu'avec la table de hachage. Il lit alors la socket et identifie les messages reçus afin d'appeler les métodes qui correspondent à la demande. La méthode \textit{actionInterested} va juste vérifier si le pair possède le fichier demandé et dans le cas affirmatif va renvoyer son \textit{buffermap} dans une réponse $have$ dans la socket.

La méthode \textit{actionGetpiece} va aussi vérifier si le pair possède le fichier. Il ouvrira ensuite ce fichier pour récupérer les parties demandées qu'il possèdera (en matchant son buffermap) avant de les envoyer dans une réponse $data$.
Cette dernière méthode n'est pas encore complètement terminée. Il faut jouer avec les indices mais la partie à récupérer est du genre $[taille\_piece*indice,taille\_piece*(indice+1)]$.


\section{Travail restant}
Mise à part la dernière fonction il reste encore du travail notamment sur la partie de demande d'information et de téléchargement.

\subsection{Thread de demande utilisateur} 
Ce thread, qui en pratique sera sûrement dans la continuité du \textit{main}, gèrera l'interface utilisateur. Cette interface basique dans un terminal devra communiquer avec le tracker pour lui demander des information sur les fichiers disponibles. Ceci dans une boucle, dans le cas où l'utilisateur veut télécharger plusieurs fichiers.

%Il est vraisemblable de dire qu'une unique socket soit nécessaire ici. 


\subsection{Gestion d'exception}
Dans le travail actuel les exceptions levées aboutissent presque toutes à l'arrêt du programme. Il est nécessaire pour la robustesse de l'application de gérer un maximum de ces exceptions. Ce travail sera fait une fois le reste fini.

%\section{Explications}
%Pourquoi cet ordre ?

