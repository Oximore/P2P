\documentclass[a4paper,11pt]{report}

\usepackage[french]{babel}
\usepackage[utf8]{inputenc}
\usepackage{graphicx}
\usepackage[colorlinks=true]{hyperref}
\hypersetup{urlcolor=blue,linkcolor=black,citecolor=black,colorlinks=true} 
%%%%%%%%%%%%%%%% Lengths %%%%%%%%%%%%%%%%
\setlength{\textwidth}{15.5cm}
\setlength{\evensidemargin}{0.5cm}
\setlength{\oddsidemargin}{0.5cm}

%%%%%%%%%%%%%%%% Variables %%%%%%%%%%%%%%%%

\title{Projet Réseau : Application d'échange de fichiers Pair à Pair}
\author{Thibaud Cheippe, Lux Benjamin, Boutin Guillaume}


\begin{document}
\maketitle



%%%%%%%%%%%%%%%% Main part %%%%%%%%%%%%%%%%
\chapter*{Introduction}

Ce projet a pour but de créer une application pair à pair. Pour cela, il faut diviser l'application en deux parties : la partie tracker et la partie pair.
Les pair sont des utilisateurs qui possèdent des fichiers à partager et/ou qui désirent télécharger des fichiers détenus par d'autres pairs. Pour cela, ils doivent se connecter au tracker : ils renseignent ainsi les fichiers qu'ils possèdent et ceux qu'ils veulent. Le tracker répond aux requêtes de téléchargement en donnant la liste des pairs possédant le fichier désiré. Grâce à ces informations, le pair peut alors commencer le téléchargement.
Le rôle du tracker est donc central, on parle d'application centralisée.

Dans ce projet, la partie paire est réalisée en Java alors que la partie tracker est réalisée en C.
\chapter{Partie Pair} 
% pair.tex

% une socket ou une sockette ?


\subsection{Généralités sur le Pair}
On dit souvent trop vite que la partie pair de ce projet joue le rôle
de client. Or même si elle a effectivement ce rôle auprès du tracker
ce n'est pas exactement le cas en général. En effet ce programme doit
aussi bien jouer le rôle de client que de serveur. Nous allons ici
détailler toutes les composantes de ce logiciel destiné à l'usage d'un
utilisateur.

\subsubsection{Implémentation Java, des classes objets}
Il a semblé pertinent de choisir le langage Java pour le pair car ce
dernier semble plus maléable et mieux réutilisable dans l'optique de
faire la partie décentralisée du projet. Malheureusement nous n'avons
pas eu le temps d'aborder cette partie du projet.

L'implémentation Java permet l'utilisation de l'héritage de classe qui
se révèle pratique lorsque deux parties différentes du programme
utilisent les mêmes mécanismes, les mêmes méthodes, mais ne sont pas
dans la même formation logique. Nous avons par exemple créé une classe
\textit{ToolsTelechargementThread} qui sera la classe mère de deux
Threads s'occupant du téléchargement, l'une utilisée par l'utilisateur,
l'autre s'occupant de la reprise de téléchargement au démarrage du
pair.

\subsubsection{Une architecture multi-thread}
Le pair, de par sa nature, est un programme fortement
multi-threadé. Cette structure multi-thread est obligatoire lorsque
l'on veut implémenter proprement un serveur, à chaque demande de
connexion celui-ci lancera un nouveau thread qui s'occupera de
répondre tandis que l'autre pourra continuer d'écouter la socket
serveur.

L'utilisation de thread est dans d'autres cas moins obligatoire mais
reste une manière élégante de gérer certains cas particulier. On peut
penser notamment aux régulières mise à jour des données avec le
serveur. Une manière basique de transmettre les données au serveur
aurait été de créer une variable contenant l'heure de la dernière mise à
jour et de la comparer régulièrement à l'heure actuelle. Cependant un
thread qui boucle avec un temps d'attente est, il me semble, beaucoup
plus propre et logique.

Nous avons même poussé le vice jusqu'à utiliser un thread pour la
partie utilisateur humain, alors que celle-ci pourrait juste être placée
à la suite de l'initialisation de toutes nos structures de
données. Cependant là encore il existe une séparation logique entre
ces deux fonctionnalités. Alors il nous a semblé plus pertinent
d'avoir aussi cette séparation dans le code.

\subsubsection{Structures de données}
% table hashage // classe de stockages (Fichier / InfoPair)
Pour pouvoir transmettre les informations utiles dans l'ensemble des
threads nous avons créé des structures de données qui même si elles
contiennent un peu trop d'information pour un thread donné, limitent le
passage d'argument et factorise le code.

La plus importante de ces structures est sans aucun doute celle qui
contient les informations sur chacun des fichiers proposés au
téléchargement disponible sur notre machine. Ces informations sont
contenues dans la classe \textit{Fichier}. On trouve dans cette classe
des attributs privés contenant le nom du fichier, sa clé, sa taille,
la taille de découpe des pièces de ce fichier et les parties de ce
fichier que l'on possède (son masque de bit). Nous y retrouvons
également des accéceurs pour lire ces variables ainsi que quelque uns
pour définir leur valeurs. 

Ces classes \textit{Fichier} sont elles-mêmes contenues dans une table
de hashage, instance de la classe native \textit{Hashtable} de Java.
Cette structure correspond exactement à ce que l'on recherche puisque
d'une part les temps de recherche sont court et d'une autre nous
pouvons utiliser les clés des fichiers à bon escient.

Il existe d'autres structures moins importantes comme
\textit{InfoPair} qui est destinée à collecter les informations sur la
liste des pairs obtenue par la requête \textit{look}.

%
%  Et le FichierConfiguration dans tout ça ?
%


\subsubsection{Communication par socket}
% communication courtes
Le protocole de communication, fixé par le sujet, est TCP/IP. Ainsi, 
la partie pair utilise les classes \textit{ServeurSocket}
et \textit{Socket} déjà implémentées en Java.

Même si le coût d'établissement d'une socket n'est pas nul, nous avons
fait le choix de privilégier les communications courtes. Ainsi après
avoir effectué un échange d'information (question/réponse)
l'initiateur de la communication coupe la socket. Même si cela peut
paraitre couteux nous en tirons deux avantages, d'une part nous
n'avons pas besoin de stocker toutes les sockets en cours
d'utilisation d'autre part nous n'avons pas à nous soucier si après un
message vient une partie de texte mal formé.


\subsection{Cheminement du programme, structuration logique}
Maintenant que nous avons survolé les principaux points de la
technique d'implémentation, nous allons suivre le déroulement du
programme dans ces actions.

\subsubsection{Programme d'initialisation}
L'utilisateur lance le programme grâce à la commande shell $$java
Pair$$ C'est donc sur la classe \textit{Pair} que nous allons nous
pencher. Celle-ci commence par charger le fichier de configuration en
instanciant un objet de la classe \textit{FichierConfiguration}. Celle-ci
 ira ouvrir le fichier \textit{configuration.txt}, puis le
parcourra afin de récupérer les valeurs insérées par l'utilisateur.

Le pair va ensuite créer un objet \textit{HashTable} et parcourir
tous les fichiers placés dans le dossier \textit{./Download}. Pour
chacun de ses fichiers, si il n'est pas un fichier caché la méthode
\textit{sauver()} sera appelée. Celle-ci instanciera un objet
\textit{Fichier} dans laquelle les informations du fichier seront
stockées à partir, soit d'un fichier caché portant le même nom, soit en
considérant que ce fichier vient d'être mis par l'utilisateur depuis
le dernier lancement du programme. On supposera que dans le second cas
le fichier est possédé en entier, on pourra donc mettre un buffermap
entier. La clé de ce fichier sera calculée à partir de la méthode
\textit{md5sum} de la classe \textit{FileHashSum}. Cette instance sera
ensuite mise dans la table de hashage avec pour élément d'entré sa
clé.

La pair lance ensuite son serveur qui est une instance de la classe
\textit{Serveur}. 

Il exécute ensuite la méthode \textit{run} de la classe
\textit{ThreadUtilisateur}.  Cette méthode ne déclenche pas
l'exécution d'un thread mais seulement du code de celui-ci. Ainsi nous
ne créons pas de thread supplémentaire à ce niveau, mais comme dit
précédement nous créons une coupure fonctionnelle.

Le pair sauve enfin les données qui ont pu être modifiées dans la table
de hashage dans des fichiers cachés cités plus haut. Puis il termine.

\subsubsection{Thread serveur}
La classe \textit{Serveur} hérite de la classe native \textit{Thread}.
Sa méthode \textit{run} consiste à créer un serveur en instanciant un
objet de la classe \textit{ServeurSocket} grâce à un port défini par
l'utilisateur en paramètre du programme, ou si l'utilisateur n'a rien
spécifié, un port aléatoire sur la machine.


Un thread de la classe \textit{MiseAJour} est ensuite créé, puis lancé avec en
argument le port précédement évoqué.

Ce thread entrera ensuite dans une boucle infinie où il attendra une
demande de connexion. Le cas échéant il créera un objet de la classe
\textit{ServeurThread} qui sera chargé de répondre au client.


\subsubsection{Thread de Mise à jour}
Ce thread crée la liste des fichiers qu'il possède, entièrement ou en
partie, puis se connecte au tracker pour se déclarer et lui indiquer
quels sont ses fichiers grâce au mot clé \textit{announce}.

Il entre ensuite dans une boucle infinie dans laquelle il attend un
certain temps (défini dans le fichier de configuration) puis se
reconnecte au tracker et se met à jour avec le mot clé \textit{update}
avant de se redéconnecter.


\subsubsection{Thread de réponse}
Ce thread appelé par le thread serveur répond aux demandes de
connexion.

Il commence d'abord par identifier le permier mot qu'on lui
transmet. Si il correspond à un mot clé \textit{interested} ou
\textit{getpiece} alors il lancera la fonction qui sera chargée de lire
la suite du message et de répondre en conséquence. Dans le cas
contraire il ignorera ce mot et continuera d'écouter le canal entrant.

Ce thread ne ferme jamais la socket, c'est au pair appelant de la
couper lorsqu'il n'aura plus rien à demander. Ainsi bien que l'on ne
demandera jamais ici deux action dans la même socket, ce thread gère
très bien ce cas (dans le cas de demandes bien formées).


\subsubsection{Thread utilisateur}
% amélioration possible : plusieurs thread pour le dl
Ce thread hérite de la classe \textit{ToolsTelechargementThread} dans
laquelle sont définies les méthodes de connexion à un serveur et les
différentes méthodes permettant d'envoyer une question à un serveur et
de recevoir sa réponse.

La méthode run de cette classe propose un menu à l'utilisateur dans
lequel ce dernier peut choisir de rechercher un fichier suivant divers
critères. Une fois les critères sélectionnés une demande \textit{look}
est envoyée au tracker. 

En fonction de la réponse du tracker une liste de fichiers répondant
aux critères est afficher à l'utilisateur, lequel devra en choisir un
ou annuler. Si l'un des fichiers est sélectionné une action
\textit{getfile} est créée.


Le thread utilisateur va ensuite demander à tous les pairs présents
dans la réponse du tracker le buffermap du fichier demandé. Après un
tri de ces buffermap plusieurs demande \textit{getpieces} seront
envoyées et les bouts de fichier reçus seront enregistrés à l'emplacement
correspondant du fichier convoité.

%\subsubsection{Reprise de téléchargement}



\chapter{Partie Tracker}

\section{Implémentation de la base de données}
\subsection{Fichiers}
Nous avons créé une base de données contenant la liste des fichiers possédés par les clients actuellement connectés. \\
Les fichiers sont représentés par une structure element. Elle contient entre autres champs un pointeur sur une chaîne de caractères \textit{key}, qui sera l'identifiant du fichier (la base pourra comporter plusiurs fichiers de même nom \textit{name} et même longueur \textit{length}, mais une clé sera unique dans la base de données). \\
Le champ \textit{peer\_list} pointe sur un objet de type \textit{struct list}, qui est ne liste chaînée de \textit{struct peer}. La structure \textit{struct peer} étant la structure que nous avons choisie pour stocker les différents utilisateurs connectés, \textit{peer\_list} représente la liste de pairs connectés possédant ce fichier.\\ 
La structure \textit{base}, qui représente notre base de données, est une liste chaînée d'\textit{element};\\
 
\begin{verbatim}

struct element
{
  struct element * next;
  char * key;
  char * name;
  int length;
  int p_size;
  struct list * peer_list; 
};

struct base
{
  struct element * first;
  int size;
};

\end{verbatim}

\subsection{Pairs}
  
Chaque pair est représenté par une structure \textit{peer}. Le champ \textit{time} permet de stocker la date de réception du dernier message d'``update'' du pair. Une autre fonction parcourera tous les pairs connectés en continu, et si la valeur de ce champ dépasse une valeur de time\_out préalablement fixée, ce pair sera supprimé.\\

\begin{verbatim}
struct peer
{
  struct peer * next;
  unsigned long ip_address;
  int port;
  int time;
  struct base * previous_update;
};

struct list
{
  struct peer * first;
  int size;
};
\end{verbatim}

Ces deux constructions sont en fait symétriques: un élément d'une des deux listes possède un champ du type de l'autre liste, ce qui laisse la possibilité de stocker les fichiers possédés dans les pairs connectés ou les pairs possédant le fichier dans les fichiers disponibles, voire les deux en stockant directement les pointeurs.\\ 
Par souci de simplicité nous avons décidé de stocker les pairs dans les fichiers.Toutefois le champ \textit{base} de la structure \textit{peer} nous sert quand même (cf Mise à jour).


\section{Les sockets}

Pour communiquer avec les pairs, le tracker doit créer des sockets. On crée aussi des threads associés à ces sockets.
\subsection{Gestion des threads et des sockets}
Pour stocker les threads et les sockets, on utilise une structure client et une structure client\_tab.

La structure client contient donc le file descriptor et la structure de la socket, ainsi que le thread associé :
\begin{verbatim}
struct client
{
  int sock;
  struct sockaddr_in *sockaddr;
  pthread_t *t; 
};
\end{verbatim}
On organise ensuite ces structures dans client\_tab :
\begin{verbatim}

struct client_tab
{
  struct client tab[MAX];
  int b[MAX];
};
\end{verbatim}
On utilise un tableau de longueur prédéfinie ainsi qu'un tableau de booléens de même taille. Ce tableau de booléens nous informe si les clients sont actifs ou déconnectés.
Nous avons choisi d'utiliser des tableaux car ils permettent de facilement limiter le nombre de connections au tracker ( pour qu'il ne soit pas surchargé ). Il est important de noter qu'un pair peut ouvrir plusieurs sockets à la suite (et donc créer plusieur structures client). Il peut par exemple ouvrir une première socket, envoyer un message, fermer la connection, puis en ouvrir une autre... Ou encore ouvrir deux sockets en parallèle et envoyer des messages sur les deux. Ce type de comportement est géré et ne pose pas de problèmes lors de la communication avec ce pair.

\subsection{Thread d'écoute}
La socket d'écoute, créée sur le premier thread, sert à attendre les connections de nouveaux pairs (il est en attente bloquante). Dès qu'un pair demande une connection, ce premier thread crée une nouvelle socket et un nouveau thread spécifique à ce pair.
Le nouveau thread est lancé avec la fonction communicate.
\subsection{Threads de communication avec les clients}
Ce thread de communication sert à recevoir et envoyer des messages au client, et ce, au moyen de la fonction communicate. Cette fonction sera décrite ci-dessous.


\section{Communication avec le client}

La fonction \textit{communicate} écoute en permanence le client et stocke les messages reçus dans un buffer. Lorsqu'on a reçu des données, on parse ce buffer et on analyse les requètes du client. On effectue les mises à jour de la base de données si necessaires, puis on renvoie un message d'information ou de confirmation au client. \\
Si il y a une erreur (ex: le client envoie un message autre que ``announce ...'' alors qu'il n'est pas dans la liste des pairs connectés), on ferme la socket et on interromp la fonction, ce qui tue le processus. Ces fermetures peuvent aussi être faites par la fonction de time out si trop de temps s'est écoulé depuis le dernier ``update''.

\section{Mise à jour de la base de données}

A la réception d'un message d'''announce'', on crée un nouveau \textit{peer} s'il n'existe pas déjà (comparaison d'adresse ip). Puis, pour chaque fichier qu'il possède, on ajoute un \textit{element} à la base de données s'il n'existe pas déjà (comparaison de clés), sinon on ajoute ce pair à la liste des pairs possédant le fichier de même clé.\\

Lors de la reception d'un message d'''update'', on met à jour le champ time du pair, puis on compare la liste des fichiers possédés par le pair à ceux du précédent update (ils étaient stockés sous forme d'une liste d'\textit{element} dans le champ \textit{previous\_update} du pair).On met alors à jour la base de données en ajoutant le pair comme possesseur des fichiers qu'il a mais n'avait pas la dernière fois, et en l'enlevant des fichiers qu'il n'a plus. Puis on place la liste des fichiers de l'update dans le champ \textit{previous\_update} du pair. 


\end{document}
